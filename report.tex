
\documentclass[12pt,titlepage]{article}

\textwidth 6.5 true in
\textheight 9.0 true in
\hoffset -.5 true in
\voffset -1.0 true in

\setlength{\topmargin}{0in}
\title{ECE 408 - Applied Parallel Programming Project}
\date{\today}
\author{
	Hwanseo Choi (hwanseo2), \\
	Simeng Liu (simengl2),\\
	Chandan Vempati (vempati2).
}

\begin{document}
\maketitle

\section*{Milestone 1}

\subsection*{Question 1}
\textbf{Team Members}: \\
	Hwanseo Choi (hwanseo2),\\
	Simeng Liu (simengl2),\\
	Chandan Vempati(vempati2).\\


\subsection*{Question 2}
The list of kernels and some their corresponding parameters which were measured is shown in Table 1.
\begin{table}[h!]
	\centering
	\begin{tabular}{||c | c | c | c||}
		\hline
		Name                                                                                                      & Time(\%) & Time          & Calls \\ [0.5ex]
		\hline\hline
		void fermiPlusCgemmLDS128  \textunderscore  batched                                                       & 34.09    & 118.48 ms     & 9     \\
		void cudnn::detail::implicit   \textunderscore   convolve  \textunderscore  sgemm                         & 27.02    & 93.917 ms     & 1     \\
		void fft2d  \textunderscore  c2r  \textunderscore  32x32                                                  & 12.62    & 43.847 ms     & 9     \\
		sgemm  \textunderscore  sm35  \textunderscore  ldg  \textunderscore  tn  \textunderscore  128x8x256x16x32 & 8.20     & 28.512 ms     & 1     \\
		CUDA  \textunderscore  memcpy  \textunderscore  HtoD                                                      & 6.46     & 22.445 ms     & 14    \\
		void cudnn::detail::activation  \textunderscore  fw  \textunderscore  4d  \textunderscore  kernel         & 4.07     & 14.158 ms     & 2     \\
		void cudnn::detail::pooling  \textunderscore  fw  \textunderscore  4d  \textunderscore  kernel            & 3.82     & 13.292 ms     & 1     \\
		void fft2d  \textunderscore  r2c  \textunderscore  32x32                                                  & 1.72     & 5.9651 ms     & 9     \\
		sgemm  \textunderscore  sm35  \textunderscore  ldg \textunderscore tn  \textunderscore  64x16x128x8x32    & 1.17     & 4.0583 ms     & 1     \\
		void mshadow::cuda::MapPlanLargeKernel                                                                    & 0.37     & 1.2844 ms     & 1     \\
		void mshadow::cuda::SoftmaxKernel                                                                         & 0.32     & 1.1046 ms     & 1     \\
		void mshadow::cuda::MapPlanKernel                                                                         & 0.05     & 177.02 $\mu$s & 13    \\
		void mshadow::cuda::MapPlanKernel                                                                         & 0.04     & 146.34 $\mu$s & 2     \\
		sgemm  \textunderscore  sm35  \textunderscore  ldg  \textunderscore  tn  \textunderscore  32x16x64x8x16   & 0.04     & 130.11 $\mu$s & 1     \\
		void mshadow::cuda::MapPlanKernel                                                                         & 0.01     & 22.399 $\mu$s & 1     \\
		void fft2d  \textunderscore  r2c  \textunderscore  32x32                                                  & 0.01     & 20.671 $\mu$s & 1     \\
		CUDA  \textunderscore  memcpy  \textunderscore  DtoH                                                      & 0.00     & 9.9200 $\mu$s & 1     \\[1ex]
		\hline
		\hline
	\end{tabular}
	\caption{CUDA kernel calls and their corresponding parameters}
\end{table}


\pagebreak
\subsection*{Question 3}

The list of APIs and some their corresponding parameters which were measured are shown in Table 2.

\begin{table}[h!]
	\centering
	\begin{tabular}{||c | c | c | c||}
		\hline
		Name                             & Time(\%) & Time          & Calls \\ [0.5ex]
		\hline\hline
		cudaStreamCreateWithFlags        & 43.55    & 1.92546 s     & 18    \\
		cudaFree                         & 27.11    & 1.19848 s     & 10    \\
		cudaMemGetInfo                   & 20.70    & 915.17 ms     & 27    \\
		cudaStreamSynchronize            & 7.31     & 323.39 ms     & 29    \\
		cudaMemcpy2DAsync                & 1.01     & 44.605 ms     & 9     \\
		cudaMalloc                       & 0.16     & 7.2049 ms     & 45    \\
		cudaStreamCreate                 & 0.03     & 1.5196 ms     & 4     \\
		cuDeviceTotalMem                 & 0.03     & 1.3522 ms     & 4     \\
		cuDeviceGetAttribute             & 0.03     & 1.1863 ms     & 352   \\
		cudaEventCreateWithFlags         & 0.02     & 1.0891 ms     & 114   \\
		cudaLaunch                       & 0.02     & 728.51 $\mu$s & 53    \\
		cudaMemcpy                       & 0.01     & 405.96 $\mu$s & 6     \\
		cudaSetupArgument                & 0.01     & 352.84 $\mu$s & 619   \\
		cudaDeviceGetAttribute           & 0.00     & 135.99 $\mu$s & 116   \\
		cuDeviceGetName                  & 0.00     & 102.82 $\mu$s & 4     \\
		cudaSetDevice                    & 0.00     & 82.812 $\mu$s & 35    \\
		cudaStreamWaitEvent              & 0.00     & 55.338 $\mu$s & 27    \\
		cudaStreamCreateWithPriority     & 0.00     & 50.246 $\mu$s & 2     \\
		cudaConfigureCall                & 0.00     & 48.487 $\mu$s & 53    \\
		cudaGetDevice                    & 0.00     & 26.507 $\mu$s & 10    \\
		cudaEventRecord                  & 0.00     & 21.586 $\mu$s & 12    \\
		cudaGetLastError                 & 0.00     & 20.909 $\mu$s & 34    \\
		cudaBindTexture                  & 0.00     & 15.628 $\mu$s & 1     \\
		cudaPeekAtLastError              & 0.00     & 12.229 $\mu$s & 18    \\
		cuDeviceGetCount                 & 0.00     & 6.8830 $\mu$s & 6     \\
		cudaEventCreate                  & 0.00     & 5.9220 $\mu$s & 1     \\
		cudaStreamGetPriority            & 0.00     & 5.8400 $\mu$s & 1     \\
		cuDeviceGet                      & 0.00     & 5.0390 $\mu$s & 6     \\
		cudaDeviceGetStreamPriorityRange & 0.00     & 4.9430 $\mu$s & 2     \\
		cuInit                           & 0.00     & 3.8850 $\mu$s & 3     \\
		cuDriverGetVersion               & 0.00     & 2.6300 $\mu$s & 3     \\
		cudaEventDestroy                 & 0.00     & 2.4120 $\mu$s & 1     \\
		cudaGetDeviceCount               & 0.00     & 2.1210 $\mu$s & 1     \\
		cudaUnbindTexture                & 0.00     & 1.8500 $\mu$s & 1     \\[1ex]
		\hline
	\end{tabular}
	\caption{CUDA APIs calls and their corresponding parameters}
\end{table}


\subsection*{Question 4}
\textbf{Application Program Interfaces} or \textbf{APIs} is a set of subroutine definitions that is provided by NVIDIA as part of the CUDA tool and is in-charge of connecting host and device for various purposes. For example, copy memory from host to device or the other direction.\\
\textbf{Kernels} are the custom code that is defined by the user and are executed $N$ times in parallel by $N$ different CUDA threads, as opposed to only once like regular C functions.


\subsection*{Question 5}
\textbf{Output of rai running MXNET on the CPU is as follows:}\\
\noindent\fbox{\parbox{\textwidth}{\texttt{
	Loading fashion-mnist data...\\
	done\\
	Loading model...\\
	done\\
	New Inference\\
	EvalMetric: {'accuracy': 0.8444}
}}}


\subsection*{Question 6}
The program run time on the CPU is \textbf{12.74 s}.


\subsection*{Question 7}
\textbf{Output of rai running MXNET on the GPU is as follows:}\\
\noindent\fbox{\parbox{\textwidth}{\texttt{
	Running /usr/bin/time python m1.2.py\\
	Loading fashion-mnist data...\\
	done\\
	Loading model...\\
	src/operator/././ cudnn \textunderscore algoreg-inl.h:112: Running performance tests to find the best convolution algorithm, this can take a while...(setting env variable MXNET\textunderscore CUDNN \textunderscore AUTOTUNE \textunderscore DEFAULT to 0 to disable)\\
	done\\
	New Inference\\
	EvalMetric: {'accuracy': 0.8444}
}}}


\subsection*{Question 8}
The program run time on the GPU is \textbf{2.13 s}.

\pagebreak
\section*{Milestone 2}

\subsection*{Question 1}

The list of total execution time for all parameters is shown in Table 3.

\begin{table}[h!]
	\centering
	\begin{tabular}{||c | c ||}
		\hline
		Number of Images & Time (s)                                      \\ [0.5ex]
		\hline\hline
		10000 (default)  & 30.58 user 1.55 system 0:30.03 elapsed        \\
		1000             & 1.06 user 0.61 system 0:01.02 elapsed         \\
		100              & 0.70 user 0.48 system 0:00.72 elapsed         \\
		\hline\hline
	\end{tabular}
	\caption{Total execution times and their corresponding parameters}
\end{table}


\subsection*{Question 2}

The list of op time for all parameters is shown in Table 4.

\begin{table}[h!]
	\centering
	\begin{tabular}{||c | c | c ||}
		\hline
		Number of Images & Op Time 1 (s) & Op Time 2 (s) \\ [0.5ex]
		\hline\hline
		10000 (default)  & 6.61          & 19.48         \\
		1000             & 0.07          &  0.20         \\
		100              & 0.01          &  0.02         \\
		\hline\hline
	\end{tabular}
	\caption{Total execution times and their corresponding parameters}
\end{table}

The Op Time scales linearly with the number of images.

\end{document}